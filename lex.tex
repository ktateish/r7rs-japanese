% Lexical structure

%%\vfill\eject
\chapter{字句規約}

この節ではSchemeのプログラムを書くために用いる字句規約の一部について
非公式な説明を述べます。
Schemeの正式な構文は\ref{BNF}~節を参照してください。

\section{識別子}
\label{syntaxsection}

識別子\mainindex{識別子}は文字、数字、「拡張識別子文字」の任意の並びです。
ただし有効な数値で始まっていてはいけません。
またリスト構文で使う \ide{.} (ピリオドひとつ)は識別子ではありません。

すべてのScheme処理系は以下の拡張識別子文字をサポートしなければなりません。

\begin{scheme}
!\ \$ \% \verb"&" * + - . / :\ < = > ? @ \verb"^" \verb"_" \verb"~" %
\end{scheme}

代わりに垂直線 ({\cf $|$}) で囲ったゼロ個以上の文字の並びで識別子を表すこともできます。
これは文字列リテラルの記法に似せたものです。
この記法ではバックスラッシュと垂直線以外の
ホワイトスペースを含む任意の文字を直接書くことができます。
さらに\meta{inline hex escape}または文字列内で利用可能なものと同じ
エスケープシーケンスを使って文字を指定することもできます。

例えば識別子 \verb+|H\x65;llo|+ は \verb+Hello+ と同じ識別子です。
また識別子 \verb+|\x3BB;|+ は識別子 $\lambda$ と同じです
(このUnicode文字をサポートしている処理系の場合)。
さらに言えば \verb+|\t\t|+ と \verb+|\x9;\x9;|+ も同じです。
ちなみに \verb+||+ も有効な識別子で、これは他のいかなる識別子とも異なります。

いくつか識別子の例を挙げます。

\begin{scheme}
...                      {+}
+soup+                   <=?
->string                 a34kTMNs
lambda                   list->vector
q                        V17a
|two words|              |two\backwhack{}x20;words|
the-word-recursion-has-many-meanings%
\end{scheme}

識別子の正式な構文は\ref{extendedalphas}~節を参照してください。

\vest Schemeのプログラムでは識別子に2つの用途があります。
\begin{itemize}
\item あらゆる識別子は変数\index{変数}として
あるいは構文キーワード\index{構文キーワード}として使うことができます
(\ref{variablesection}~節および\ref{macrosection}~節を参照)。

\item 識別子がリテラル(\ref{quote}~節を参照)としてあるいはリテラル内に現れることで
{\em シンボル}(\ref{symbolsection}~節を参照)を表します。
\end{itemize}

以前のバージョンの報告書\cite{R5RS}と異なり、
識別子および文字の名前で使われる大文字小文字は区別されます。
しかし識別子、文字、文字列の構文の中の\meta{inline hex escape}や数値では
大文字小文字は区別されません。
この報告書で定義されている識別子に大文字を含むものはありません。

以下の指令で大文字小文字の区別を明示的に制御できます。

\begin{entry}{%
{\cf{}\#!fold-case}\sharpbangindex{fold-case}\\
{\cf{}\#!no-fold-case}\sharpbangindex{no-fold-case}}

これらの指令はコメントを書ける場所(\ref{wscommentsection}~節を参照)ならどこにでも書くことができます。
指令の後にはデリミタを置かなければなりません。
指令はコメントとして扱われます。
ただし同じポートから読み込む後続のデータに影響を与えます。
{\cf{}\#!fold-case}指令は後続の識別子と文字名の大文字小文字を区別しないようにします。
これは{\cf string-foldcase}(\ref{stringsection}~節を参照)を適用したかのように扱われます。
文字リテラルには影響しません。
{\cf{}\#!no-fold-case}指令は大文字小文字を区別するデフォルトの動作に戻します。
\end{entry}



\section{ホワイトスペースとコメント}
\label{wscommentsection}

\defining{ホワイトスペース}には空白、タブ、改行といった文字が含まれます。
(処理系は改ページのような追加のホワイトスペース文字を提供しても構いません。)
ホワイトスペースは可読性を上げるため
必要に応じてトークン同士を分離するために使われますが、
それ以外には意味の無いものです
(トークンとは識別子や数値のような分割できない字句単位のことです)。
ホワイトスペースは2つのトークンの間に置くことはできますが、
ひとつのトークンの途中に置くことはできません。
文字列の中や垂直線で区切られたシンボルの中に現れるホワイトスペースは意味のあるものです。

字句構文にはコメントの形式がいくつかあります。
コメントはホワイトスペースとまったく同様に扱われます。

セミコロン ({\tt;}) は一行コメントの開始を表します。\mainindex{コメント}\mainschindex{;}
このコメントはセミコロンが現れた行の終わりまで続きます。

コメントを表すもうひとつの方法は\hyper{datum}
(\ref{datumsyntax}~節を参照)
の前に {\tt \#;}\sharpindex{;} を付けることです。
オプショナルな\meta{whitespace}を挟むこともできます。
このコメントはコメント接頭辞 {\tt \#;}、空白、そして\hyper{datum}から成ります。
この記法はコードの一部を「コメントアウト」するのに便利です。

ブロックコメントは %
{\tt \#|}\schindex{\#|}\schindex{|\#}
および {\tt |\#} の組で表します。
これはネストできます。

\begin{scheme}
\#|
   The FACT procedure computes the factorial
   of a non-negative integer.
|\#
(define fact
  (lambda (n)
    (if (= n 0)
        \#;(= n 1)
        1        ;Base case: return 1
        (* n (fact (- n 1))))))%
\end{scheme}


\section{その他の表記}

\todo{Rewrite?}

数値に使われる記法の説明は\ref{numbersection}~節を参照してください。

\begin{description}{}{}

\item[{\tt.\ + -}]
これらは数値に使われますが、識別子の中の任意の場所でも使うことができます。
単独のプラス記号およびマイナス記号もまた識別子です。
単独のピリオド(数値や識別子の中に現れたものでない)はペアの表記に使われます(\ref{listsection}~節)。
また仮引数リストの残余引数を表すためにも使われます(\ref{lambda}~節)。
ちなみに2個以上のピリオドの並びは識別子{\em です}。

\item[\tt( )]
括弧はグループ化のためやリストを表すために使われます
(\ref{listsection}~節)。

\item[\singlequote]
アポストロフィ(シングルクォート)文字はリテラルデータを表すために使われます(\ref{quote}~節)。

\item[\backquote]
グレーブアクセント(バッククォート)文字は部分的に定数であるデータを表すために使われます
(\ref{quasiquote}~節)。

\item[\tt, ,@]
コンマおよびコンマ・アットマークの並びはquasiquoteの中で使われます(\ref{quasiquote}~節)。

\item[\tt"]
ダブルクォート文字は文字列を区切るために使われます(\ref{stringsection}~節)。

\item[\backwhack]
バックスラッシュは文字定数(\ref{charactersection}~節)の構文で使われます。
また文字列定数(\ref{stringsection}~節)および識別子(\ref{extendedalphas}~節)の中で
エスケープ文字としても使われます。

% A box used because \verb is not allowed in command arguments.
\setbox0\hbox{\tt \verb"[" \verb"]" \verb"{" \verb"}"}
\item[\copy0]
角括弧および波括弧は将来の言語の拡張の可能性のために予約されています。

\item[\sharpsign]
井桁は直後の文字によって様々な目的に使われます。

\item[\schtrue{} \schfalse{}]
これらはブーリアン定数です(\ref{booleansection}~節)。
さらに \sharpfoo{true}および \sharpfoo{false}も同様です。

\item[\sharpsign\backwhack]
これは文字定数の始まりです(\ref{charactersection}~節)。

\item[\sharpsign\tt(]
これはベクタ定数の始まりです(\ref{vectorsection}~節)。
ベクタ定数は~{\tt)}~で終わります。

\item[\sharpsign\tt u8(]
これはバイトベクタ定数の始まりです(\ref{bytevectorsection}~節)。
バイトベクタ定数は~{\tt)}~で終わります。

\item[{\tt\#e \#i \#b \#o \#d \#x}]
これらは数値を表すために使われます(\ref{numbernotations}~節)。

\item[\tt{\#\hyper{n}= \#\hyper{n}\#}]
これらは他のリテラルデータの参照やラベル付けに使われます(\ref{labelsection}~節)。

\end{description}

\section{データムラベル}\unsection
\label{labelsection}

\begin{entry}{%
\pproto{\#\hyper{n}=\hyper{datum}}{字句構文}
\pproto{\#\hyper{n}\#}{字句構文}}

字句構文 \texttt{\#\hyper{n}=\hyper{datum}} は\hyper{datum}と同様に解釈されますが、
その結果の\hyper{datum}に\hyper{n}でラベルを付けます。
\hyper{n}が数字の並びでない場合はエラーです。

字句構文 \texttt{\#\hyper{n}\#} は \texttt{\#\hyper{n}=}
でラベルを付けたオブジェクトへの参照として機能します。
\texttt{\#\hyper{n}}= と同じオブジェクトが結果となります
(\ref{equivalencesection}~節を参照)。

これらの構文を使うと部分的な共有構造や循環構造を表すことができます。

\begin{scheme}
(let ((x (list 'a 'b 'c)))
  (set-cdr! (cddr x) x)
  x)                       \ev \#0=(a b c . \#0\#)
\end{scheme}

データムラベルのスコープはそれが現れた最も外側のデータムのうちそのラベルより右側の部分です。
従って \texttt{\#\hyper{n}\#} はラベル \texttt{\#\hyper{n}=} より後にだけ現れることができます。
前方参照を試みることはエラーです。
またラベルを付けたオブジェクトそれ自身として参照が現れた場合はエラーです。
例えば \texttt{\#\hyper{n}= \#\hyper{n}\#} のような場合です。
この場合 \texttt{\#\hyper{n}=} がラベルを付けているオブジェクトが何であるか不明なためです。

\hyper{program}または\hyper{library}にリテラル以外の循環参照がある場合はエラーです。
特に{\cf quasiquote}(\ref{quasiquote}~節)に循環参照がある場合はエラーです。

\begin{scheme}
\#1=(begin (display \#\backwhack{}x) \#1\#)
                       \ev \scherror%
\end{scheme}
\end{entry}


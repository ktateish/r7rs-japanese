\chapter{標準の特徴識別子}
\label{stdfeatures}

処理系は{\cf cond-expand}および{\cf features}で用いるために
以下の一覧にある特徴識別子の一部またはすべてを提供しても構いません。
ただし、特徴が提供されていない場合は、
それに対応する特徴識別子も提供してはなりません。

\label{standard_features}

\feature{r7rs}{すべての \rsevenrs\ Scheme処理系はこの特徴を持つ。}
\feature{exact-closed}{{\cf /}を除くすべての代数演算は
正確な入力が与えられると正確な値を生成する。}
\feature{exact-complex}{正確な複素数が提供されている。}
\feature{ieee-float}{不正確な数値は IEEE 754 二進浮動小数点の値である。}
\feature{full-unicode}{Unicodeバージョン6.0に存在するすべての文字が
Schemeの文字としてサポートされている。}
\feature{ratios}{正確な引数で{\cf /}を呼ぶと
除数がゼロでなければ正確な結果を生成する。}
\feature{posix}{この処理系はPOSIXシステム上で実行されている。}
\feature{windows}{この処理系はWindows上で実行されている。}
\feature{unix, darwin, gnu-linux, bsd, freebsd, solaris, ...}{
オペレーティングシステムのフラグ(おそらくひとつ以上)。}
\feature{i386, x86-64, ppc, sparc, jvm, clr, llvm, ...}{
CPUアーキテクチャのフラグ。}
\feature{ilp32, lp64, ilp64, ...}{Cのメモリモデルのフラグ。}
\feature{big-endian, little-endian}{バイトオーダーのフラグ。}
\feature{\hyper{name}}{処理系の名前。}
\feature{\hyper{name-version}}{処理系の名前およびバージョン。}

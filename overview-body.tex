\chapter*{Schemeの概要}

この文章は \rsevenrs の小さな言語の概要を記載しています。
この概要の目的は、リファレンスマニュアルとして編成されている
\rsevenrs{} の理解を手助けするために、言語の基本概念について十分な説明をすることです。
その目的のため、この概要は言語の完全な入門書ではありませんし、
すべての事柄や基準について、いかなる意味でも正確に述べているわけではありません。

\vest Algolに倣い、Schemeは静的なスコープを持つプログラミング言語です。
変数はその変数の字句的に見えている束縛に関連付けられます。

\vest Schemeは暗黙の型を持つ言語です。
これは明示的な型と反対の概念です。
型は変数ではなくオブジェクト (値とも呼ばれます) に紐付けられます。
(暗黙な型を持つ言語のことを、型を持たない言語、弱い型を持つ言語、動的な型を持つ言語などと呼ぶ人もいます。)
暗黙な型を持つ言語は他にPython、Ruby、Smalltalkなどがあります。
他のLisp方言もそうです。
明示的な型を持つ言語
(強い型を持つ言語や静的な型を持つ言語と呼ばれることもあります)
にはAlgol 60、C、C\#、Java、Haskell、MLなどがあります。

\vest Schemeの計算中に作成されるすべてのオブジェクトは無制限の存続期間を持ちます。
手続きや継続もそれに含まれます。
Schemeのオブジェクトは破棄されることはありません。
Scheme処理系が(通常は!)記憶領域を使い切ることが無いのは、
あるオブジェクトが将来のいかなる計算にも影響を与える可能性がないと保証できる場合に、
そのオブジェクトが占有している記憶領域を回収することが出来るためです。
ほとんどのオブジェクトが無制限の存続期間を持つ言語には、
C\#、Java、Haskell、ほとんどのLisp方言、ML、Python、Ruby、Smalltalkなどがあります。

Scheme処理系は、真正末尾再帰を行うことが要求されます。
これにより、繰り返し計算が構文的には再帰手続きとして記述されていても、
その繰り返し計算を一定の空間内で実行することが可能になります。
処理系が真正末尾再帰を行うことにより、
通常の手続き呼び出しを用いて繰り返しを表現することができます。
そのため、特殊な繰り返し構文は構文糖衣としての価値しかありません。

\vest Schemeはオブジェクトとしての手続きをサポートした最初の言語のひとつです。
手続きは動的に作成したり、データ構造に格納したり、
手続きの結果として返したりすることが出来ます。
このような特徴を持つ言語にはCommon Lisp、Haskell、ML、Ruby、Smalltalkなどがあります。

\vest Scheme特有の機能のひとつに、第一級の地位を持つ継続があります。
これは、他のほとんどの言語では、水面下にしか存在しないものです。
第一級の継続を利用することで、非局所脱出、バックトラッキング、コルーチンなど、
幅広い様々な高度な制御構造を実装出来ます。

Schemeでは、手続き呼び出しの引数の式は、手続きが制御を得る前に、
必要とされるか否かに関わらず、評価されます。
C、C\#、Common Lisp、Python、Ruby、Smalltalkなどが手続き呼び出しの前に必ず引数式を評価する言語の例です。
これは、手続きが必要としない限り引数が評価されないHaskellの遅延評価の意味論や
Algol 60の名前渡しの意味論とはまったく異なるものです。

Schemeの数値計算モデルには豊富な数値型とその演算子が用意されています。
さらに、\textit{正確}な数値と\textit{不正確}な数値を区別しています。
原則として、正確な数値オブジェクトは正確にあるひとつの数値に対応するもので、
不正確な数値は丸めやその他の近似を伴った計算の結果です。

\chapter{基本的な型}

Schemeのプログラムは\textit{オブジェクト}を操作します(\textit{値}と呼ばれることもあります)。
Schemeのオブジェクトは\textit{型}と呼ばれる値の集合に分類されます。
この章ではScheme言語の基礎となる重要な型の概要を述べます。

\begin{note}
  Schemeは暗黙の型を持つので、Schemeの文脈における用語\textit{型}の用途は、
  他の言語、特に明示的な型を持つ言語における文脈での用途とは異なります。
\end{note}

\paragraph{数値}

Schemeは幅広い豊富な数値データ型をサポートしています。
任意精度の整数、有理数、複素数や様々な型の不正確な数値などがあります。

\paragraph{ブーリアン}

ブーリアンは真偽値です。
真または偽のいずれかとなります。
Schemeでは、``偽''のオブジェクトは \schfalse{} と書きます。
``真''のオブジェクトは \schtrue{} と書きます。
しかし、真偽値を必要とする場所のほとんどで、
\schfalse{} 以外のあらゆるオブジェクトが真とみなされます。

\paragraph{ペアとリスト}

ペアは2つの部分を持つデータ構造です。
ペアのほとんどの用途は(単方向連結)リストを表現することです。
ひとつめの部分(``car'')がリストの最初の要素を表し、
ふたつめの部分(``cdr'')がリストの残りを表します。
Schemeにはさらに独立した空リストがあり、
リストを形成するペアのチェーンの最後の cdr となります。

\paragraph{シンボル}

シンボルは文字列を表すオブジェクトです。
シンボルが表す文字列は、そのシンボルの\textit{名前}と呼ばれます。
文字列オブジェクトと異なり、同じ綴りの名前を持つ2つのシンボルは区別されません。
シンボルには多くの活用方法があります。
例えば、他の言語では列挙型を使うような場面で、シンボルを使うことができます。

\rfivers と異なり、 \rsevenrs ではシンボルおよび識別子の大文字小文字を区別します。

\paragraph{文字}

Schemeの文字は文章の文字にほぼ対応します。
より正確にいうと、Unicode標準の\textit{スカラ値}のサブセットと同じです。
処理系によってはさらに拡張されている場合もあります。

\paragraph{文字列}

文字列は有限の文字の並びで、固定の長さを持ちます。
すなわち、任意のUnicodeテキストを表します。

\paragraph{ベクタ}

ベクタは、リストのように、任意のオブジェクトの有限の並びを表す線形のデータ構造です。
リストの要素はそれを表しているペアのチェーンを辿って線形にアクセスされるのに対し、
ベクタの要素は整数のインデックスでアクセスされます。
そのため、要素にランダムアクセスする場合は、リストよりベクタの方が適しています。

\paragraph{バイトベクタ}

バイトベクタはベクタに似ていますが、要素が\textit{バイト}である点が異なります。
つまり、0~255の範囲の正確な整数しか格納できません。

\paragraph{手続き}

Schemeでは手続きは値です。

\paragraph{レコード}

レコードは構造化された値であり、ゼロ個以上の\textit{フィールド}の集合体です。
それぞれのフィールドにひとつずつ場所があります。
レコードは\textit{レコード型}に分類されます。
述語、構築手続き、フィールド参照手続き、フィールド変更手続きが
それぞれのレコード型に対して定義されます。

\paragraph{ポート}

ポートは入出力機器を表します。
Schemeでは、入力ポートは命令に応じてデータを供給するSchemeオブジェクトで、
出力ポートはデータを消費するSchemeオブジェクトです。

\chapter{式}

Schemeコードの最も重要な要素が\textit{式}です。
式は、\textit{評価}して\textit{値}を生成することができます。
最も基礎的な式はリテラル式です:

\begin{scheme}
\schtrue{} \ev \schtrue
23 \ev 23%
\end{scheme}

この表記は式 \schtrue{} が \schtrue{}、つまり``真''を表す値に評価され、
式 {\cf 23} が23という数を表す数値に評価されることを意味しています。

合成式は部分式のまわりに括弧を置くことで作られます。
最初の部分式は演算を区別し、残りの部分式はその演算の被演算子です。
%
\begin{scheme}
(+ 23 42) \ev 65
(+ 14 (* 23 42)) \ev 980%
\end{scheme}
%
最初の例では、{\cf +}は組み込みの加算演算子の名前で、{\cf 23}と{\cf 42}がその被演算子です。
式 {\cf (+ 23 42)} は``23と42の和''と解釈します。
合成式は入れ子にできます ---
ふたつめの例は``23と42の積と14の和''と解釈します。

これらの例が示すように、Schemeの合成式は常に同じ前置表記を使って書きます。
必然的に構造を示すために括弧が必要となります。
そのため、数学の表記や多くのプログラミングで許容されている``不要な''括弧は、Schemeでは許容されません。

多くのプログラミング言語と同様に、
それが式の部分式を分割している場合、
ホワイトスペース(改行も含みます)は重要ではなく、
構造を表すために使うことができます。

\chapter{変数と束縛}

Schemeでは識別子を用いて値を保持する場所を表すことができます。
これらの識別子は変数と呼ばれます。
多くの場合において、特にその場所の値が作成後に変更されることがない場合、
変数はその値を直接表すものとして考えるのが便利でしょう。

\begin{scheme}
(let ((x 23)
      (y 42))
  (+ x y)) \ev 65%
\end{scheme}

この場合、{\cf let}で始まる式は束縛構文です。
{\cf let}の次の括弧で囲まれた部分は、変数と式のリストです。
変数{\cf x}と{\cf 23}、そして変数{\cf y}と{\cf 42}です。
{\cf let}式は{\cf x}を23に束縛し、{\cf y}を42に束縛します。
これらの束縛は{\cf let}式の\textit{本体}の中、つまり {\cf (+ x y)} でのみ有効です。

\chapter{定義}

{\cf let}式で束縛された変数は\textit{局所的}です。
それらの束縛は{\cf let}の本体からのみ見えます。
以下のようにして識別子に対する最上位の束縛を作ることもできます。

\begin{scheme}
(define x 23)
(define y 42)
(+ x y) \ev 65%
\end{scheme}

(これらは最上位のプログラムまたはライブラリの本体内で実際に``最上位''です。)

最初の2つの括弧で囲まれた部分は\textit{定義}です。
これらは{\cf x}を23に、{\cf y}を42に束縛する最上位の束縛を作成します。
定義は式ではありません。
式を書けるすべての場所に定義が書けるわけではありません。
また、定義は値を持ちません。

束縛はプログラムの字句的構造に従います。
同じ名前の束縛がいくつか存在する場合、変数は最も近い束縛を参照します。
プログラム中のその変数が現れた場所から始まり、内側から外側へと進み、
その途中で局所的な束縛が見付からなければ、
最も外側の束縛を参照します。

\begin{scheme}
(define x 23)
(define y 42)
(let ((y 43))
  (+ x y)) \ev 66

(let ((y 43))
  (let ((y 44))
    (+ x y))) \ev 67%
\end{scheme}

\chapter{手続き}

定義を使って手続きを定義することもできます。

\begin{scheme}
(define (f x)
  (+ x 42))

(f 23) \ev 65%
\end{scheme}

手続きは、いくらか単純化した、オブジェクトに対する式の抽象化です。
この例では、ひとつめの定義は{\cf f}という名前の手続きを定義しています。
{\cf f x}のまわりの括弧は、これが手続き定義であることを表しています。
式 {\cf (f 23)} は手続きの呼び出しで、だいたい
``{\cf x} を23に束縛して {\cf (+ x 42)} (手続きの本体です)を評価する''
のような意味合いです。

手続きはオブジェクトなので、他の手続きに渡すことができます。
%
\begin{scheme}
(define (f x)
  (+ x 42))

(define (g p x)
  (p x))

(g f 23) \ev 65%
\end{scheme}

この例では、{\cf p}を{\cf f}に束縛し、{\cf x}を23に束縛して、{\cf g}の本体が評価されます。
これは {\cf (f 23)} と等価であり、すなわち65に評価されます。

実のところ、Schemeの定義済み手続きの多くは、
構文としてではなく、値が手続きである変数として用意されています。
例えば{\cf +}演算子は、
他の多くの言語では特別な構文として扱われていますが、
Schemeでは普通の識別子です。
単に数値を加算する手続きに束縛されているだけです。
{\cf *}や他の多くの演算子も同様です。

\begin{scheme}
(define (h op x y)
  (op x y))

(h + 23 42) \ev 65
(h * 23 42) \ev 966%
\end{scheme}

手続き定義は手続きを作る唯一の方法ではありません。
{\cf lambda}式を使うと、名前を指定する必要なしに、
オブジェクトとしての新しい手続きを作ることができます。

\begin{scheme}
((lambda (x) (+ x 42)) 23) \ev 65%
\end{scheme}

この例の式全体は手続き呼び出しです。
{\cf (lambda (x) (+ x 42))} が数値をひとつ取りそれに42を加算する手続きに評価されます。

\chapter{手続き呼び出しと構文キーワード}

{\cf (+ 23 42)} や {\cf (f 23)}、
{\cf ((lambda (x) (+ x 42))} などがすべて手続き呼び出しの例であるのに対し、
{\cf lambda}式や{\cf let}式はそうではありません。
これは{\cf let}が、識別子ではあるものの、変数ではなく\textit{構文キーワード}であるためです。
最初の部分式に構文キーワードを持つリストは、そのキーワードによって決まる特別な規則に従います。
定義で使われる{\cf define}識別子も構文キーワードです。
従って、定義は手続き呼び出しではありません。

{\cf lambda}キーワードの規則では、
最初の部分リストは引数リストで、残りの部分リストは手続きの本体になります。
{\cf let}式の場合は、
最初の部分リストは束縛指定のリストで、残りの部分リストは式の本体を構成します。

こういった\textit{式型}と手続き呼び出しとは、
リストの最初の位置に構文キーワードを見つけることで区別されます。
もし最初の位置に構文キーワードがなければ、その式は手続き呼び出しです。
Schemeの構文キーワードはとても少なく、この作業は実に簡単になっています。
しかしながら、新しい構文キーワードに対する束縛を作ることもできます。

\chapter{代入}

定義や{\cf let}、{\cf lambda}で束縛したSchemeの変数は、
実際にはそれぞれの束縛で指定されたオブジェクトに直接束縛されるわけではありません。
そうではなく、それらのオブジェクトを保持している場所に束縛されます。
これらの場所の内容は、その後に\textit{代入}によって破壊的に変更できます。
%
\begin{scheme}
(let ((x 23))
  (set! x 42)
  x) \ev 42%
\end{scheme}

この例では、{\cf let}式の本体には2つの式があります。
これらは順番に評価され、最後の式の値が{\cf let}式全体の値となります。
式 {\cf (set! x 42)} は代入です。
``{\cf x}が参照している場所のオブジェクトを42に置き換えなさい''という意味です。
つまり、{\cf x}の以前の値23は、42に置き換えられます。

\chapter{派生構文とマクロ}

\rsevenrs{} の小さな言語の一部として規定されている式型の多くは、
より基本的な式型に変換できます。
例えば、{\cf let}式は手続き呼び出しと{\cf lambda}式に変換できます。
以下のふたつの式は等価です:
%
\begin{scheme}
(let ((x 23)
      (y 42))
  (+ x y)) \ev 65

((lambda (x y) (+ x y)) 23 42) \lev 65%
\end{scheme}

{\cf let}式のような構文は\textit{派生構文}と呼ばれます。
その意味論が構文変換により他の形の式から派生しているためです。
手続きにも派生式として定義されているものがあります。
以下のふたつの式は等価です:

\begin{scheme}
(define (f x)
  (+ x 42))

(define f
  (lambda (x)
    (+ x 42)))%
\end{scheme}

Schemeでは、構文キーワードをマクロに束縛することで、
プログラム自身が独自の派生式を作ることができます。

\begin{scheme}
(define-syntax def
  (syntax-rules ()
    ((def f (p ...) body)
     (define (f p ...)
       body))))

(def f (x)
  (+ x 42))%
\end{scheme}

この{\cf define-syntax}式は、
パターン {\cf (def f (p ...) body)} に一致する括弧で囲まれた構文を
{\cf (define (f p ...) body)} に変換する、という指定です。
{\cf f}、{\cf p}、{\cf body}はパターン変数です。
従って、この例の{\cf def}式は、以下のように変換されます:

\begin{scheme}
(define (f x)
  (+ x 42))%
\end{scheme}

新しい構文キーワードを作る能力により、
Schemeは非常に柔軟で表現力が高い言語となっています。
他の言語で組み込まれている多くの機能はSchemeで直接実装することができます。
Schemeプログラマなら誰でも、新しい式型を追加できるのです。

\chapter{構文datumとdatum値}

\textit{datum値}はSchemeオブジェクトのサブセットを構成します。
ブーリアン、数値、文字、シンボル、文字列、
および要素がdatum値であるリスト、ベクタ、バイトベクタがこれに含まれます。
それぞれのdatum値は\textit{構文datum}としてテキストで表現することができ、
情報の損失なしに書き出し、読み戻すことが出来ます。
それぞれのdatum値に対応する構文datumは一般的にひとつ以上あります。
さらに、それぞれのdatum値は対応する構文datumに
{\cf\singlequote} を前置することでプログラム内のリテラル式に変換することができます。

\begin{scheme}
'23 \ev 23
'\schtrue{} \ev \schtrue{}
'foo \ev foo
'(1 2 3) \ev (1 2 3)
'\#(1 2 3) \ev \#(1 2 3)%
\end{scheme}

この例のうち、シンボルとリスト以外のリテラル定数表現には {\cf\singlequote} は不要です。
構文datum {\cf foo} は名前が``foo''であるシンボルを表し、
{\cf 'foo} はそのシンボルを値として持つリテラル式です。
{\cf (1 2 3)} は要素が1、2、3であるリストを表す構文datumであり、
{\cf '(1 2 3)} はそのリストを値として持つリテラル式です。
同様に、
{\cf \#(1 2 3)} は要素が1、2、3であるベクタを表す構文datumであり、
{\cf '\#(1 2 3)} は対応するリテラルです。

構文datumはSchemeの式のスーパーセットです。
すなわち、データを使ってSchemeの式をデータオブジェクトとして表現することができます。
特に、シンボルを使って識別子を表現できます。

\begin{scheme}
'(+ 23 42) \ev (+ 23 42)
'(define (f x) (+ x 42)) \lev (define (f x) (+ x 42))%
\end{scheme}

これは、Schemeのソースコードを操作するプログラム、
特にインタプリタやプログラム変換器を書く基礎となります。

\chapter{継続}

Schemeの式を評価するときは常に、その式の結果を欲している\textit{継続}が存在しています。
継続はその計算の(デフォルトの)未来全体を表現します。
例えば、以下の式の{\cf 3}の継続は、
%
\begin{scheme}
(+ 1 3)%
\end{scheme}
%
それに1を加算することです。
通常、これらの普遍的に存在している継続は、水面下に隠されており、
プログラマはそれらについて考えていません。
しかし、プログラマが明示的に継続を扱う必要のある状況がまれにあります。
現在の継続を復元する手続きを作成する
{\cf call-with-current-continuation}
手続きにより、Schemeプログラマは継続を扱うことが出来ます。

以下の例では、引数に1を加算する継続を表す脱出手続きを{\cf escape}に束縛し、
引数として3を与えてそれを呼び出しています。
{\cf escape}への呼び出しの継続は放棄され、代わりに1を加算する継続に3が渡されます。
%
\begin{scheme}
(+ 1 (call-with-current-continuation
       (lambda (escape)
         (+ 2 (escape 3))))) \lev 4%
\end{scheme}
%
脱出手続きは無制限の存続期間を持ちます。
その継続を補足した呼び出しが戻った後に呼び出すことができ、
複数回呼び出すことも出来ます。
これにより {\cf call-with-current-continuation}
は他の言語の例外処理のような典型的な非局所制御構文に比べて非常に強力なものになっています。

\chapter{ライブラリ}

Schemeのコードは\textit{ライブラリ}と呼ばれる部品にまとめることができます。
ライブラリは定義と式を持ち、
他のライブラリから定義をインポートしたり、
他のライブラリに定義をエクスポートしたりできます。

以下に示す {\cf (hello)} という名前のライブラリは、
{\cf hello-world}という名前の定義をエクスポートし、
baseライブラリとdisplayライブラリをインポートしています。
エクスポートしている{\cf hello-world}は、
{\cf Hello World}を表示して改行する手続きです。
%
\begin{scheme}
(define-library (hello)
  (export hello-world)
  (import (scheme base)
          (scheme display))
  (begin
    (define (hello-world)
      (display "Hello World")
      (newline))))%
\end{scheme}

\chapter{プログラム}

ライブラリは他のライブラリから、最終的にはSchemeの\textit{プログラム}から呼び出されます。
ライブラリ同様に、プログラムはインポート、定義、式を持つことができ、実行の開始点を規定します。
そのためプログラムは、ライブラリインポートの推移閉包によって、Schemeのプログラムを定義します。

以下のプログラムは、
process-contextライブラリの{\cf command-line}手続きを使って、
コマンドラインから最初の引数を取得します。
それから{\cf with-input-from-file}を使ってファイルを開きます。
この手続きはファイルを現在の入力ポートとし、最後に閉じるように調整します。
次に、{\cf read-line}手続きを呼んでファイルからテキストを1行読み込み、
そして{\cf write-string}および{\cf newline}でその行を出力し、
それをファイルの終端まで繰り返します。
%
\begin{scheme}
(import (scheme base)
        (scheme file)
        (scheme process-context))
(with-input-from-file
  (cadr (command-line))
  (lambda ()
    (let loop ((line (read-line)))
      (unless (eof-object? line)
        (write-string line)
        (newline)
        (loop (read-line))))))%
\end{scheme}

\chapter{REPL}

処理系によっては \defining{REPL} (Read-Eval-Print Loop)
と呼ばれる対話環境を用意しているものがあります。
インポート宣言、式、定義を、一度にひとつずつ入力し、評価できる環境です。
REPLはbaseライブラリをインポートした状態で開始されます。
他のライブラリをインポートしている場合もあります。
REPLにファイルから入力を読み込むモードを用意している処理系もあります。
そういったファイルは、開始以外の場所にインポート宣言を持つことができ、
一般的にプログラムと同じではありません。

以下に短いREPLの対話の様子を示します。
文字 {\cf >} はREPLの入力プロンプトを表しています。

\begin{scheme}
> ; A few simple things
> (+ 2 2)
4
> (sin 4)
Undefined variable: sin
> (import (scheme inexact))
> (sin 4)
-0.756802495307928
> (define sine sin)
> (sine 4)
-0.756802495307928
> ; Guy Steele's three-part test
> ; True is true ...
> \#t
\#t
> ; 100!/99! = 100 ...
> (define (fact n)
    (if (= n 0) 1 (* n (fact (- n 1)))))
> (/ (fact 100) (fact 99))
100
> ; If it returns the *right* complex number,
> ; so much the better ...
> (define (atanh x)
    (/ (- (log (+ 1 x))
          (log (- 1 x)))
       2))
> (atanh -2)
-0.549306144334055+1.5707963267949i%
\end{scheme}

%%% Local Variables: 
%%% mode: latex
%%% End: 

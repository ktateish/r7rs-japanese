\chapter{標準ライブラリ}
\label{stdlibraries}

%% Note, this is used to generate stdmod.tex.  The bindings could be
%% extracted automatically from the document, but this lets us choose
%% the ordering and optionally format manually where needed.

この節では標準ライブラリによって提供されるエクスポートの一覧を掲載しています。
ライブラリは処理系によってはサポートされない可能性のある機能や
ロードに時間がかかる機能を分離するように編成されています。

すべての標準ライブラリに対して{\cf scheme}ライブラリ接頭辞が使われています。
またこれは将来の標準で使用するために予約されています。

\textbf{baseライブラリ}

\texttt{(scheme base)}ライブラリは
伝統的にSchemeと紐付けられている多数の手続きと構文の束縛をエクスポートしています。
baseライブラリと他の標準ライブラリは構造ではなく用途によって分けられています。
特に、
他の標準手続きや構文に基づいた機能ではなく
一般的にコンパイラやランタイムシステムによりプリミティブとして実装されている機能には
baseライブラリの一部ではなく別のライブラリで定義されているものもあります。
同様に、baseライブラリのエクスポートには
他のエクスポートに基づいて実装できるものもあります。
これらは厳密に言うと冗長ではあるものの、共通の使用パターンを捕えており、
そのため便利な短縮形として用意されています。

\begin{scheme}
._               ...
.*                +                -
./                <=               <
.=>               =                >=
.>                abs              and
.append           apply            assoc
.assq             assv             begin
.boolean?          boolean=?       bytevector
.bytevector-copy  bytevector-append  bytevector-copy!
.bytevector-length bytevector-u8-ref bytevector-u8-set!
.bytevector?            caar             cadr
.call-with-current-continuation     call-with-values
.call-with-port
.call/cc          case
.car              cdr
.cdar    cddr
.ceiling          char->integer    char<=?
.char<?
.char=?           char>=?          char>?
.char?            complex?         cond
.cond-expand
.cons             define-syntax    define
.define-values
.define-record-type                 denominator
.do               dynamic-wind     else
.eq?
.equal?           eqv?             error
.error-object?    error-object-message  error-object-irritants
.even?            exact            exact-integer-sqrt
.exact-integer?   exact?           expt
.floor            for-each         gcd
.floor/     floor-quotient     floor-remainder
.truncate/  truncate-quotient  truncate-remainder
.features         guard            if
.include          include-ci
.inexact          inexact?
.integer->char    integer?         lambda
.lcm              length           let*
.let-syntax       letrec*          letrec-syntax
.let-values       let*-values
.letrec           let              list-copy
.list->string     list->vector     list-ref
.list-set!        list-tail        list?
.list             make-bytevector  make-list
.make-parameter   make-string      make-vector
.map              max              member
.memq             memv             min
.modulo           negative?        not
.null?            number->string   number?
.numerator        odd?             or
.pair?            parameterize
.positive?
.procedure?       quasiquote       quote
.quotient         raise-continuable
.raise            rational?        rationalize
.real?            remainder        reverse
.round            set!             set-car!
.set-cdr!         square
.string->list     string->number
.string->symbol   string->vector   string-append
.string-copy      string-copy!
.string-fill!     string-for-each
.string-length    string-map       string-ref
.string-set!      string<=?        string<?
.string=?         string>=?        string>?
.string?          string           substring
.symbol->string   symbol=?
.symbol?          syntax-error
.syntax-rules     truncate         values
.unquote          unquote-splicing
.vector-append    vector-copy      vector-copy!
.vector->list     vector->string   vector-fill!
.vector-for-each  vector-length    vector-map
.vector-ref       vector-set!      vector?
.vector           zero?            when
.with-exception-handler            unless
.binary-port?             char-ready?
.textual-port?            close-port
.close-input-port
.close-output-port        current-error-port
.current-input-port       current-output-port
.eof-object               eof-object?
.file-error?              flush-output-port
.get-output-string        get-output-bytevector
.input-port?              input-port-open?
.newline
.open-input-string        open-input-bytevector
.open-output-string       open-output-bytevector
.output-port?             output-port-open?
.peek-char
.peek-u8                  port?
.read-bytevector          read-bytevector!
.read-char                read-error?
.read-line                read-string
.read-u8                  string->utf8
.utf8->string             u8-ready?
.write-bytevector         write-char
.write-string             write-u8
\end{scheme}

\textbf{case-lambdaライブラリ}

\texttt{(scheme case-lambda)}ライブラリは{\cf case-lambda}構文を
エクスポートしています。

\begin{scheme}
.case-lambda
\end{scheme}

\textbf{charライブラリ}

\texttt{(scheme char)}ライブラリには
Unicode全体をサポートすると巨大なテーブルが必要になる可能性がある
文字を扱う手続きが用意されています。

\begin{scheme}
.char-alphabetic?
.char-ci=?       char-ci<?       char-ci>?
.char-ci<=?      char-ci>=?      char-upcase
.char-downcase   char-foldcase   char-lower-case?
.char-numeric?   char-upper-case?
.char-whitespace?                 string-ci=?
.string-ci<?     string-ci>?     string-ci<=?
.string-ci>=?    string-upcase   string-downcase
.string-foldcase
.digit-value
\end{scheme}

\textbf{complexライブラリ}

\texttt{(scheme complex)}ライブラリは
一般的に実数でない数値でのみ役に立つ手続きをエクスポートしています。

\begin{scheme}
.angle   magnitude   imag-part   real-part
.make-polar           make-rectangular
\end{scheme}

\textbf{cxrライブラリ}

\texttt{(scheme cxr)}ライブラリは
{\cf car}および{\cf cdr}を3つまたは4つ合成した
24個の手続きをエクスポートしています。
例えば{\cf caddar}は以下のように定義出来ます。

\begin{scheme}
(define caddar
  (lambda (x) (car (cdr (cdr (car x)))))){\rm.}%
\end{scheme}

手続き{\cf car}および{\cf cdr}自身、
およびそれらの2段合成である4個の手続きは
baseライブラリに含まれています。
\ref{listsection}~節を参照してください。

\begin{scheme}
.caaaar caaadr caadar caaddr
.cadaar cadadr caddar cadddr
.cdaaar cdaadr cdadar cdaddr
.cddaar cddadr cdddar cddddr
.caaar caadr cadar caddr
.cdaar cdadr cddar cdddr
\end{scheme}

\textbf{evalライブラリ}

\texttt{(scheme eval)}ライブラリは
Schemeのデータをプログラムとして評価するための手続きをエクスポートしています。

\begin{scheme}
.eval
.environment
\end{scheme}

\textbf{fileライブラリ}

\texttt{(scheme file)}ライブラリには
ファイルにアクセスするための手続きが用意されています。

\begin{scheme}
.call-with-input-file    call-with-output-file
.delete-file             file-exists?
.open-input-file         open-output-file
.open-binary-input-file  open-binary-output-file
.with-input-from-file    with-output-to-file
\end{scheme}

\textbf{inexactライブラリ}

\texttt{(scheme inexact)}ライブラリは
一般的に不正確な値を扱う場合にのみ役に立つ手続きをエクスポートしています。

\begin{scheme}
.acos      asin atan
.cos       exp  finite?
.infinite? log  nan?
.sin       sqrt tan
\end{scheme}

\textbf{lazyライブラリ}

\texttt{(scheme lazy)}ライブラリは
遅延評価のための手続きと構文キーワードをエクスポートしています。

\begin{scheme}
.delay   delay-force   force   make-promise   promise?
\end{scheme}

\textbf{loadライブラリ}

\texttt{(scheme load)}ライブラリは
ファイルからSchemeの式をロードする手続きをエクスポートしています。

\begin{scheme}
.load
\end{scheme}

\textbf{process-contextライブラリ}

\texttt{(scheme process-context)}ライブラリは
プログラムを呼び出している文脈にアクセスするための手続きをエクスポートしています。

\begin{scheme}
.get-environment-variable
.get-environment-variables
.command-line
.emergency-exit
.exit
\end{scheme}

\textbf{readライブラリ}

\texttt{(scheme read)}ライブラリには
Schemeのオブジェクトを読み取る手続きが用意されています。

\begin{scheme}
.read
\end{scheme}

\textbf{replライブラリ}

\texttt{(scheme repl)}ライブラリは
{\cf interaction-environment}手続きをエクスポートしています。

\begin{scheme}
.interaction-environment
\end{scheme}

\textbf{timeライブラリ}

\texttt{(scheme time)}ライブラリには
時間に関係する値へのアクセスが用意されています。

\begin{scheme}
.current-second
.current-jiffy
.jiffies-per-second
\end{scheme}

\textbf{writeライブラリ}

\texttt{(scheme write)}ライブラリには
Schemeのオブジェクトを書き出すための手続きが用意されています。

\begin{scheme}
.write  write-shared write-simple  display
\end{scheme}

\textbf{r5rsライブラリ}

\texttt{(scheme r5rs)}ライブラリには
\rfivers で定義されている識別子が用意されています。
ただし{\cf transcript-on}および{\cf transcript-off}はありません。
{\cf exact}および{\cf inexact}手続きはそれぞれ
その \rfivers での名前である
{\cf inexact->exact}および{\cf exact->inexact}として現れることに注意してください。
ただし、処理系が特定のライブラリ(例えばcomplexライブラリ)を用意していない場合は、
それに対応する識別子はこのライブラリにも現れません。

\begin{scheme}
.- * / + < <= = > >= abs acos and angle append apply asin assoc assq
.assv atan begin boolean?
.caaaar caaadr caadar caaddr
.cadaar cadadr caddar cadddr
.cdaaar cdaadr cdadar cdaddr
.cddaar cddadr cdddar cddddr
.caaar caadr cadar caddr
.cdaar cdadr cddar cdddr
.caar cadr cdar cddr
.call-with-current-continuation call-with-input-file call-with-output-file
.call-with-values car case cdr ceiling char->integer char-alphabetic?
.char-ci<? char-ci<=? char-ci=? char-ci>? char-ci>=? char-downcase
.char-lower-case? char-numeric? char-ready? char-upcase
.char-upper-case? char-whitespace? char? char<? char<=? char=? char>?
.char>=? close-input-port close-output-port complex? cond cons cos
.current-input-port current-output-port define define-syntax delay
.denominator display do dynamic-wind eof-object? eq? equal? eqv? eval
.even? exact->inexact exact? exp expt floor for-each force gcd if
.imag-part inexact->exact inexact? input-port? integer->char integer?
.interaction-environment lambda lcm length let let-syntax let* letrec
.letrec-syntax list list->string list->vector list-ref list-tail list?
.load log magnitude make-polar make-rectangular make-string make-vector
.map max member memq memv min modulo negative? newline not
.null-environment null? number->string number? numerator odd?
.open-input-file open-output-file or output-port? pair? peek-char
.positive? procedure? quasiquote quote quotient rational? rationalize
.read read-char real-part real? remainder reverse round
.scheme-report-environment set-car! set-cdr! set! sin sqrt string
.string->list string->number string->symbol string-append string-ci<?
.string-ci<=? string-ci=? string-ci>? string-ci>=? string-copy
.string-fill! string-length string-ref string-set! string? string<?
.string<=? string=? string>? string>=? substring symbol->string symbol?
.tan truncate values vector vector->list vector-fill! vector-length
.vector-ref vector-set! vector? with-input-from-file
.with-output-to-file write write-char zero?
\end{scheme}

%\vfill\eject
\chapter{基本概念}
\label{basicchapter}

\section{変数、構文キーワード、有効範囲}
\label{specialformsection}
\label{variablesection}

識別子\index{identifier}は構文の種類または値が格納される場所に名前を付けるために使われます。
構文の種類に名前を付ける識別子は{\em 構文キーワード}\mainindex{syntactic keyword}と呼ばれ、
その構文の変換器に{\em 束縛されている}と言います。
場所に名前を付ける識別子は{\em 変数}\mainindex{variable}と呼ばれ、
その場所に{\em 束縛されている}と言います。
プログラム内のある地点で見える有効な束縛\mainindex{binding}すべての集合は、
その地点での有効な{\em 環境}と呼ばれます。
ある変数が束縛されている場所に格納されている値は、その変数の値と呼ばれます。
しばしば用語の誤った使い方により、
変数が値に名前を付けているだとか、変数が値に束縛されているだとか言われることがあります。
これらはまったく正確ではありませんが、この慣習が混乱を招くことはほとんどないでしょう。

\vest 新しい種類の構文を作成し、その構文に構文キーワードを束縛するには、ある特定の形式を使います。
また、新しい場所を作成し、その場所に変数を束縛するには、また別の形式を使います。
これらの形式は{\em 束縛構文}と呼ばれます。
\mainindex{binding construct}
構文キーワードを束縛する種類の式は\ref{macrosection}~節に記載されています。
最も基礎的な変数の束縛構文は{\cf lambda}式です。
他の変数束縛構文はすべて{\cf lambda}式を使って説明できます。
他の変数束縛構文には{\cf let}, {\cf let*}, {\cf letrec},
{\cf letrec*}, {\cf let-values}, {\cf let*-values},
および{\cf do}式があります
(\ref{lambda}~節, \ref{letrec}~節, \ref{do}~節を参照)。

%Note: internal definitions not mentioned here.

\vest Schemeはブロック構造を持つ言語です。
プログラム内で束縛した各々の識別子には、
その束縛が見えるプログラムテキストの\defining{有効範囲}が対応しています。
有効範囲はその束縛を確立した束縛構文の種類によって決まります。
例えば{\cf lambda}式によって確立した束縛の場合、
その有効範囲はその{\cf lambda}式全体です。
識別子の使用は、その識別子の束縛のうち、その使用場所を含んでいる最も内側の有効範囲を確立したものを参照します。
その使用場所を含んでいる有効範囲を持つその識別子の束縛がない場合、
大域環境の変数に対する束縛があれば、それを参照します。
その識別子に対する束縛がなければ、
その識別子は\defining{束縛されていない}と言います。\mainindex{bound}\index{globalenvironment}

\section{型の独立性}
\label{disjointness}

あらゆるオブジェクトは、以下の述語を2つ以上満たすことはありません。

\begin{scheme}
boolean?          bytevector?
char?             eof-object?
null?             number?
pair?             port?
procedure?        string?
symbol?           vector?
\end{scheme}

また、{\cf define-record-type}で作成された述語もすべてこれに含まれます。

オブジェクトの型
({\em ブーリアン、バイトベクタ、文字}、空リストオブジェクト、
{\em EOFオブジェクト、数値、ペア、ポート、手続き、文字列、シンボル、ベクタ}
およびすべてのレコード型)は、これらの述語により定義されます。
\mainindex{type}\schindex{boolean?}\schindex{pair?}\schindex{symbol?}
\schindex{number?}\schindex{char?}\schindex{string?}\schindex{vector?}
\schindex{bytevector?}\schindex{port?}\schindex{procedure?}\index{empty list}
\schindex{eof-object?}

独立したブーリアン型が存在してはいますが、
条件判定目的においてはすべてのSchemeの値をブーリアン値として使うことが出来ます。
\ref{booleansection}~で説明しているように、
条件判定では\schfalse{}以外のすべての値が真とみなされます。
この報告書では、``真''という言葉は\schfalse{}以外のすべてのSchemeの値を表し、
``偽''という言葉は\schfalse{}を表します。\mainindex{true} \mainindex{false}

\section{外部表現}
\label{externalreps}

Scheme(およびLisp)における重要な概念にオブジェクトの{\em 外部表現}があります。
外部表現は、そのオブジェクトを表す文字の並びです。
例えば、整数28の外部表現は `{\tt 28}'' という文字の並びです。
また整数8と整数13から成るリストの外部表現は ``{\tt(8 13)}'' という文字の並びです。

オブジェクトの外部表現は唯一であるとは限りません。
整数28は ``{\tt \#e28.000}'' や ``{\tt\#x1c}'' のようにも表せますし、
前段落のリストは ``{\tt( 08 13)}'' や ``{\tt(8 .\ (13 .\ ()))}'' のようにも表現出来ます
(\ref{listsection}~節を参照)。

多くのオブジェクトには標準の外部表現がありますが、
手続きなどいくつかのオブジェクトには、標準の外部表現がありません
(処理系によってはそういったオブジェクトにも外部表現を定義している場合があります)。

外部表現は、プログラム内で、対応するオブジェクトを得るために使うことが出来ます
(\ref{quote}~節, {\cf quote}を参照)。

外部表現は入出力のために使うことも出来ます。
手続き{\cf read}(\ref{read}~節)は外部表現を解析し、
手続き{\cf write}(\ref{write}~節)は外部表現を生成します。
これらはエレガントでパワフルな入出力機能です。

ちなみに ``{\tt(+ 2 6)}'' という文字の並びは、
シンボル{\tt +}と整数2および整数6から成る3要素のリストの外部表現であることに注意してください。
これは整数8に評価される式ではありますが、整数8の外部表現ではありません。
Schemeの構文には、式を表す文字の並びが何らかのオブジェクトの外部表現でもある、という特徴があります。
これは混乱を招くこともあります。
ある文字列の並びがデータを表しているのかプログラムを表しているのか、
文脈なしには判断できない場合があるためです。
しかしこれはパワーの源でもあります。
インタプリタやコンパイラのような、
プログラムをデータとして扱う(またはその逆の)プログラムを書くのが簡単になります。

様々な種類のオブジェクトの外部表現の構文は、
\ref{initialenv}章の各節にそのオブジェクトを扱うプリミティブの記述と一緒に
記載されています。

\section{記憶領域のモデル}
\label{storagemodel}

ペア、文字列、ベクタ、バイトベクタといったオブジェクトや変数は、
場所\mainindex{location}または場所の並びを暗黙に持ちます。
例えば文字列は、その文字列の長さと同じだけの数の場所を持ちます。
{\tt string-set!}手続きを使うと、これらの場所のひとつに新しい値を格納できます。
しかし、その文字列の持つ場所が以前と異なる場所になるわけではありません。

変数参照や{\cf car}、{\cf vector-ref}、{\cf string-ref}といった手続きによって
ある場所から取得したオブジェクトは、
その場所に最後に格納されたオブジェクトと\ide{eqv?}の意味で等価です
(\ref{equivalencesection}~節)。

すべての場所には使用中かどうかを示す印が付けられています。
変数やオブジェクトが使用中でない場所を参照することはありません。

変数やオブジェクトに対して記憶領域が新たに割り当てられると言うとき、
それは使用中でない場所から適切な数の場所を選び、
使用中であることを示す印をその場所に付け、
そしてその変数またはオブジェクトにその場所を与える、
ということを意味しています。
これとは異なり、空リストは新たに割り当てることは出来ません。
唯一のオブジェクトであるためです。
場所を持たない空文字列、空ベクタ、空バイトベクタは、
新たに割り当てることが出来る場合もあれば出来ない場合もあります。

場所を持つすべてのオブジェクトは、
可変\index{mutable}または不変\index{immutable}のいずれかです。
リテラル定数および\ide{symbol->string}から返される文字列は不変オブジェクトです。
{\cf scheme-report-environment}から返される環境は不変の場合があります。
この報告書に掲載されているそれ以外の手続きから返されるオブジェクトはすべて可変です。
不変オブジェクトの持つ場所に新しい値を格納しようとすることはエラーです。

%% If an implementation makes it impossible for any program to alter an
%% immutable object, it may treat the object as a value (similar to a
%% number, boolean, symbol, or the empty list) rather than as a container
%% for immutable locations.
\todo{Shinn: The previous comment that it is an error to mutate these
  locations, combined with the following comment that they are only
  conceptual, makes this note seem unecessary.}

これらの場所は、概念的なものであり、物理的なものではないと解釈されます。
つまり、メモリアドレスに対応付けられている必要はなく、
対応付けられている場合でもそのメモリアドレスが固定されている必要はありません。

\begin{rationale}
多くのシステムにおいて、
定数\index{constant}(例えばリテラル式の値)は読み込み専用メモリに置くのが好ましいと考えられます。
定数の変更をエラーとすることにより、
他のシステムに可変オブジェクトと不変オブジェクトの区別を要求することなく、
そういった実装戦略を取ることが出来るようになります。
\end{rationale}

\section{真正末尾再帰}
\label{proper tail recursion}

Schemeの処理系は{\em 真正末尾再帰}\mainindex{proper tail recursion}を行うことが要求されます。
後述する特定の構文文脈で行われる手続き呼び出しを{\em 末尾呼び出し}と呼びます。
真正末尾再帰を行うとは、実行中の末尾呼び出しの数に制限が無いということです。
呼び出しが{\em 実行中}であるとは、その呼び出された手続きがまだ戻っていないということです。
ちなみに、
後に呼び出される現在の継続または{\cf call-with-current-continuation}によって
予め捕捉した継続によって戻る呼び出しもこれに含まれます。
もし継続が捕捉されていなければ、すべての呼び出しは多くとも1回だけ戻ることができ、
実行中の呼び出しとは、そのうちまだ戻っていない呼び出しのことになります。
真正末尾再帰の正式な定義は\cite{propertailrecursion}~にあります。

\begin{rationale}

直感的に言って、実行中の末尾呼び出しは空間を消費しません。
末尾呼び出しで使われる継続は、その呼び出しを含む手続きに渡される継続と同じ意味論を持つためです。
不正な処理系では、その呼び出しに新しい継続を使用し、
この新しい継続へ戻った直後にその手続きに渡された継続へ戻るかもしれません。
真正末尾再帰を行う処理系は、その継続に直接戻ります。

真正末尾再帰はSteeleとSussmanの元々のSchemeの中核となるアイディアのひとつです。
彼らの最初のSchemeインタプリタには関数とアクタの両方を実装していました。
制御の流れはアクタで表現されていました。
関数は呼び出し元に結果を返しますが、アクタはそれと異なり、結果を他のアクタに渡すものでした。
この節の用語で言うと、それぞれのアクタが終了するときは他のアクタに末尾呼び出しをしていました。

SteeleとSussmanは後に、
彼らのインタプリタ内のアクタを処理するコードと関数を処理するコードが
同一のものであることに気付きました。
つまり、言語に両方を含める必要は無かったのです。

\end{rationale}

{\em 末尾呼び出し}\mainindex{tail call}は{\em 末尾文脈}で行われる手続き呼び出しです。
末尾文脈は帰納的に定義されます。
末尾文脈は常に特定のラムダ式について決定されることに注意してください。

\begin{itemize}
\item 以下の\hyper{tail expression}で示されるように、ラムダ式の本体の最後の式は末尾文脈です。
{\cf case-lambda}式のすべての本体についても同様です。
\begin{grammar}%
(l\=ambda \meta{formals}
  \>\arbno{\meta{definition}} \arbno{\meta{expression}} \meta{tail expression})

(case-lambda \arbno{(\meta{formals} \meta{tail body})})
\end{grammar}%

\item 以下の式のいずれにおいても、それが末尾文脈にあれば、
\meta{tail expression}で示される部分式は末尾文脈です。
これらは\ref{formalchapter}~章に記載されている文法規則から派生したもので、
\meta{body}のいくつかを\meta{tail body}に、
\meta{expression}のいくつかを\meta{tail expression}に、
\meta{sequence}のいくつかを\meta{tail sequence}に置き換えています。
ここに示されているのは、それらの規則のうち末尾文脈を含むもののみです。

\begin{grammar}%
(if \meta{expression} \meta{tail expression} \meta{tail expression})
(if \meta{expression} \meta{tail expression})

(cond \atleastone{\meta{cond clause}})
(cond \arbno{\meta{cond clause}} (else \meta{tail sequence}))

(c\=ase \meta{expression}
  \>\atleastone{\meta{case clause}})
(c\=ase \meta{expression}
  \>\arbno{\meta{case clause}}
  \>(else \meta{tail sequence}))

(and \arbno{\meta{expression}} \meta{tail expression})
(or \arbno{\meta{expression}} \meta{tail expression})

(when \meta{test} \meta{tail sequence})
(unless \meta{test} \meta{tail sequence})

(let (\arbno{\meta{binding spec}}) \meta{tail body})
(let \meta{variable} (\arbno{\meta{binding spec}}) \meta{tail body})
(let* (\arbno{\meta{binding spec}}) \meta{tail body})
(letrec (\arbno{\meta{binding spec}}) \meta{tail body})
(letrec* (\arbno{\meta{binding spec}}) \meta{tail body})
(let-values (\arbno{\meta{mv binding spec}}) \meta{tail body})
(let*-values (\arbno{\meta{mv binding spec}}) \meta{tail body})

(let-syntax (\arbno{\meta{syntax spec}}) \meta{tail body})
(letrec-syntax (\arbno{\meta{syntax spec}}) \meta{tail body})

(begin \meta{tail sequence})

(d\=o \=(\arbno{\meta{iteration spec}})
  \>  \>(\meta{test} \meta{tail sequence})
  \>\arbno{\meta{expression}})

{\rm ただし}

\meta{cond clause} \: (\meta{test} \meta{tail sequence})
\meta{case clause} \: ((\arbno{\meta{datum}}) \meta{tail sequence})

\meta{tail body} \: \arbno{\meta{definition}} \meta{tail sequence}
\meta{tail sequence} \: \arbno{\meta{expression}} \meta{tail expression}
\end{grammar}%

\item
{\cf cond}式または{\cf case}式が末尾文脈であり、それが
{\cf (\hyperi{expression} => \hyperii{expression})} 形式の節を持っている場合、
\hyperii{expression}の評価結果である手続きの(暗黙の)呼び出しは末尾文脈です。
\hyperii{expression}自身は末尾文脈ではありません。


\end{itemize}

この報告書で定義されている一部の手続きも、末尾呼び出しを行うことが要求されています。
\ide{apply}の第1引数、
\ide{call-with-current-continuation}の第1引数、および
\ide{call-with-values}の第2引数は、末尾呼び出しで呼び出されなければなりません。
同様に、\ide{eval}の第1引数は、
\ide{eval}内の末尾位置で呼び出されたかように評価されなければなりません。

以下の例では、{\cf f}の呼び出しはすべて末尾呼び出しです。
{\cf g}および{\cf h}の呼び出しは末尾呼び出しではありません。
{\cf x}の参照は末尾文脈ですが、呼び出しではないので末尾呼び出しではありません。
\begin{scheme}%
(lambda ()
  (if (g)
      (let ((x (h)))
        x)
      (and (g) (f))))
\end{scheme}%

\begin{note}
処理系によっては、上記の例の{\cf h}のような、末尾呼び出しでない呼び出しの一部を、
末尾呼び出しとして評価可能であることを認識できる場合があります。
上記の例では、{\cf let}式を{\cf h}の末尾呼び出しとしてコンパイルすることが出来ます。
({\cf h}が複数の値を返す可能性は無視できます。
その場合{\cf let}の効果は明示的に規定されていないので、処理系依存です。)
\end{note}


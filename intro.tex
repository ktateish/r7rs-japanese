\clearextrapart{導入}

\label{historysection}

プログラミング言語は機能の上に機能を積み重ねるのではなく、
追加の機能が必要とならないように弱点と制限を取り除くことによって設計するべきです。
組み合わせ方に制限さえなければ、
式を構成する非常に少数の規則だけで
今日使われる主なプログラミングパラダイムのほとんどをサポートするのに
十分柔軟で実用的かつ効率的なプログラミング言語を
十分に形作れます。
Schemeがそれを実証しています。

Scheme
はラムダ計算で用いられるような第一級の手続きを取り入れた最初のプログラミング言語のひとつです。
それにより動的な型を持つ言語において静的スコープ規則およびブロック構造が有用であることを示しました。
Schemeはラムダ式とシンボルから手続きを独立させ、すべての変数に単一の字句環境を用い、
手続き呼び出しにおける演算子の位置を被演算子の位置と同じように評価する最初の主要なLisp方言です。
Schemeは繰り返しを表現するために全面的に手続き呼び出しに依存することにより、
手続きの末尾呼び出しが本質的には引数を渡すGOTOであるという事実を強調しました。
それにより一貫性と効率性を両立するプログラミングスタイルが可能となりました。
Schemeは第一級の脱出手続きを採用した最初の広く使われたプログラミング言語です。
これによりそれまで知られていた逐次実行の制御構造はすべて合成することができるようになりました。
後のバージョンでは正確な数値および不正確な数値が導入されました。
これはCommon Lispの汎用算術の拡張です。
さらに近年ではSchemeは健全なマクロをサポートした最初のプログラミング言語になりました。
これにより一貫性があり信頼できる方法でブロック構造言語の構文を拡張できます。
\todo{Ramsdell:
I would like to make a few comments on presentation.  The most
important comment is about section organization.  Newspaper writers
spend most of their time writing the first three paragraphs of any
article.  This part of the article is often the only part read by
readers, and is important in enticing readers to continue.  In the
same way, The first page is most likely to be the only page read by
many SIGPLAN readers.  If I had my choice of what I would ask them to
read, it would be the material in section 1.1, the Semantics section
that notes that scheme is lexically scoped, tail-recursive, weakly
typed, ... etc.  I would expand on the discussion on continuations,
as they represent one important difference between Scheme and other
languages.  The introduction, with its history of scheme, its history
of scheme reports and meetings, and acknowledgments giving names of
people that the reader will not likely know, is not that one page I
would like all to read.  I suggest moving the history to the back of
the report, and use the first couple of pages to convince the reader
that the language documented in this report is worth studying.
}

\subsection*{背景}

\vest Schemeの最初の記述は1975年に書かれました~\cite{Scheme75}。
報告書の改定版~\cite{Scheme78}は1978年に出され、
MITによる実装が革新的なコンパイラをサポートするようアップグレードすると共に
言語の進化が述べられました~\cite{Rabbit}。
1981年および1982年にMIT、イェール大学、およびインディアナ大学の授業で
Schemeの亜種を用いるための3つの異なるプロジェクトが始められました~\cite{Rees82,MITScheme,Scheme311}。
1984年にはSchemeを用いたコンピュータサイエンスの入門用の教科書が出版されました~\cite{SICP}。

\vest Schemeはさらに広まり、ローカルな方言が枝分かれし始め、
学生と研究者がお互いに書いたコードを理解するのがしばしば難しくなってきました。
そのため、より優れた、より広く受け入れられるSchemeの標準を作るために、
1984年の10月、15人の主要なScheme処理系の代表者が集まりました。
彼らの報告書RRRS~\cite{RRRS}は1985年の夏にMITおよびインディアナ大学で出版されました。
1986年の春にさらなる改定が加えられ、\rthreers~\cite{R3RS}が作られました。
1988年の春の作業では\rfourrs~\cite{R4RS}が作られ、
これは1991年のSchemeプログラミング言語のIEEE標準~\cite{IEEEScheme}の基礎となりました。
1998年には高水準の健全なマクロ、複数の戻り値、
およびevalなどいくつかの追加要素がIEEE標準に加えられ、
\rfivers~\cite{R5RS}としてまとめられました。

\todo{Perhaps flesh out a little and mention the forming of the
 Steering Committee.}

2006年の秋、さらなる野心的な標準を作る作業が始められました。
それには数多くの新たな改良と移植性の向上に焦点を置いたより厳密な要求事項が含まれていました。
その成果である標準\rsixrs{}は2007年の8月に完成し~\cite{R6RS}、
言語のコアと必須の標準ライブラリの集合として編纂されました。
これに準拠した新たなScheme処理系がいくつも作られました。
しかしながら既存の\rfivers{}処理系(実質的にメンテナンスされていないものを除いても)
はほとんど\rsixrs{}を採用せず、またはその一部だけを選択的に採用したに過ぎませんでした。

その結果、2009年の8月、Scheme標準化委員会は標準を別々の、しかし互換性のある2つの言語
---「小さな」言語と「大きな」言語に分割する決定を下しました。
前者は教育者や研究者、埋め込み言語のユーザなどに適したもので、
\rfivers{}との互換性に焦点を置いています。
後者は主流のソフトウェア開発における実用的なニーズに焦点を置いたもので、
\rsixrs{}の置き換えとなることを意図しています。
この報告書はそのうちの「小さな」言語について記述したものです。
そのためこれを単独で\rsixrs{}の後継と見なすことはできません。



\medskip

この報告書はSchemeコミュニティ全体に属することを意図しています。
そのため料金不要で全体または一部を複写する許可が与えられています。
特にScheme処理系の作成者がそのマニュアルや他のドキュメントを作る開始点として
この報告書を使うことを推奨しています。
必要に応じて修正を加えても構いません。




\subsection*{謝辞}

支援と助言を戴いた標準化委員会のメンバー
William Clinger, Marc Feeley, Chris Hanson, Jonathan Rees, および Olin Shivers
に感謝の意を表します。

この報告書はコミュニティの多大な努力の成果であり、
以下の人々を含めコメントやフィードバックを戴いたすべての人に感謝の意を表します:
David Adler, Eli Barzilay, Taylan Ulrich
Bay\i{}rl\i/Kammer, Marco Benelli, Pierpaolo Bernardi,
Peter Bex, Per Bothner, John Boyle, Taylor Campbell, Raffael Cavallaro,
Ray Dillinger, Biep Durieux, Sztefan Edwards, Helmut Eller, Justin
Ethier, Jay Reynolds Freeman, Tony Garnock-Jones, Alan Manuel Gloria,
Steve Hafner, Sven Hartrumpf, Brian Harvey, Moritz Heidkamp, Jean-Michel
Hufflen, Aubrey Jaffer, Takashi Kato, Shiro Kawai, Richard Kelsey, Oleg
Kiselyov, Pjotr Kourzanov, Jonathan Kraut, Daniel Krueger, Christian
Stigen Larsen, Noah Lavine, Stephen Leach, Larry D. Lee, Kun Liang,
Thomas Lord, Vincent Stewart Manis, Perry Metzger, Michael Montague,
Mikael More, Vitaly Magerya, Vincent Manis, Vassil Nikolov, Joseph
Wayne Norton, Yuki Okumura, Daichi Oohashi, Jeronimo Pellegrini, Jussi
Piitulainen, Alex Queiroz, Jim Rees, Grant Rettke, Andrew Robbins, Devon
Schudy, Bakul Shah, Robert Smith, Arthur Smyles, Michael Sperber, John
David Stone, Jay Sulzberger, Malcolm Tredinnick, Sam Tobin-Hochstadt,
Andre van Tonder, Daniel Villeneuve, Denis Washington, Alan Watson,
Mark H.  Weaver, G\"oran Weinholt, David A. Wheeler, Andy Wingo, James
Wise, J\"org F. Wittenberger, Kevin A. Wortman, Sascha Ziemann

さらに過去の編集者や彼らを手助けしたすべての人に感謝の意を表します:
Hal Abelson, Norman Adams, David
Bartley, Alan Bawden, Michael Blair, Gary Brooks, George Carrette,
Andy Cromarty, Pavel Curtis, Jeff Dalton, Olivier Danvy, Ken Dickey,
Bruce Duba, Robert Findler, Andy Freeman, Richard Gabriel, Yekta
G\"ursel, Ken Haase, Robert Halstead, Robert Hieb, Paul Hudak, Morry
Katz, Eugene Kohlbecker, Chris Lindblad, Jacob Matthews, Mark Meyer,
Jim Miller, Don Oxley, Jim Philbin, Kent Pitman, John Ramsdell,
Guillermo Rozas, Mike Shaff, Jonathan Shapiro, Guy Steele, Julie
Sussman, Perry Wagle, Mitchel Wand, Daniel Weise, Henry Wu, および Ozan
Yigit.
Scheme 311 バージョン 4 のリファレンスマニュアルから文章の拝借を許可いただいた
Carol Fessenden, Daniel Friedman, および Christopher Haynes に感謝します。
{\em TI Scheme Language Reference Manual}~\cite{TImanual85}
の文章を許可いただいた Texas Instruments, Inc.~に感謝します。
影響を受けた MIT Scheme~\cite{MITScheme}, T~\cite{Rees84},
Scheme 84~\cite{Scheme84}, Common Lisp~\cite{CLtL},
および Algol 60~\cite{Naur63} のマニュアルと共に、
{\cf http://srfi.schemers.org} で公開されている
SRFI 0, 1, 4, 6, 9, 11, 13, 16, 30, 34, 39, 43, 46, 62, および 87
に感謝の意を表します。

%% \vest We also thank Betty Dexter for the extreme effort she put into
%% setting this report in \TeX, and Donald Knuth for designing the program
%% that caused her troubles.

%% \vest The Artificial Intelligence Laboratory of the
%% Massachusetts Institute of Technology, the Computer Science
%% Department of Indiana University, the Computer and Information
%% Sciences Department of the University of Oregon, and the NEC Research
%% Institute supported the preparation of this report.  Support for the MIT
%% work was provided in part by
%% the Advanced Research Projects Agency of the Department of Defense under Office
%% of Naval Research contract N00014-80-C-0505.  Support for the Indiana
%% University work was provided by NSF grants NCS 83-04567 and NCS
%% 83-03325.
